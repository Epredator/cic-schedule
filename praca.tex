\documentclass[polish,12pt,oneside]{mwbk}

% marginesy
\usepackage[a4paper,includeheadfoot,inner=3.0cm,outer=2.5cm,top=3cm,bottom=3cm]{geometry}

%\setlength\overfullrule{5pt}    % nie wiem co to
\linespread{1.1} %trochę powietrza
\clubpenalty=10000 %nie dieli wyrazów pomiędzy stronami 
%\widowpenalty=10000 %nie pozostawia wdów i sierot pojedynczych 

% usawienia nagłówków stron
\usepackage{fancyhdr}
\pagestyle{fancy}
\setlength{\headsep}{2em}
\renewcommand{\chaptermark}[1]{\markboth{\thechapter.\ #1}{}}
\fancyhf{}
\fancyhead[LE,RO]{\thepage}
\fancyhead[LO,RE]{\textit{\nouppercase{\leftmark}}}

% do składu matematki
%\usepackage{amsmath}
%\usepackage{amsfonts}

% pseudokod jak u Rivesta i Cormena
%\usepackage{newalg}
%\newcommand{\algnazwa}[1]{\textsc{#1}}
%\newcommand{\varnazwa}[1]{$#1$}

% przyzwoity listing kodu
%\usepackage{listings}
%\lstset{language=perl,basicstyle=\tiny,frame=lines,emphstyle=\underbar,numberstyle=\tiny,numbersep=5pt}

% do includowania grafiki
\usepackage[pdftex]{graphicx}
% najogólniejsze środowisko na rysunki
\newenvironment{specrysunek}%
{\begin{figure}\centering}%
{\end{figure}}
% wstawia rysunek i o ile nie podano skali do zmniejsza o połowę
\newcommand{\wstawrysunek}[2][scale=0.5]{\includegraphics[#1]{#2}}
% all-in-one wstawiające rysunek, dające etykietkę i podpis
\newcommand{\rysunek}[4][scale=0.5]{\begin{specrysunek}\wstawrysunek[#1]{#2}\caption{#3}\label{#4}\end{specrysunek}}

\usepackage[utf8]{inputenc}       % bo piszę w utf-ie
\usepackage[polish,french]{babel} % języki, które będą w pracy (do dzielenia wyrazów)
\usepackage{polski}               % ale zasadniczo to po polsku będzie

%\usepackage[all]{xy}       % potwór do rysowania
%\usepackage{subfigure}     % rysunki w rysunkach
%\usepackage[table]{xcolor} % normalne nazwy kolorów
%\usepackage{tabularx}      % lepsze tabelki
%\usepackage{verbatim}      % <pre>
\usepackage{hyperref}       % odnośniki w pdf-ie
%\usepackage{natbib}        % ?
\bibliographystyle{plplain} % normalna zwykła bibliografia
%\hyphenation{graphviz}     % wyrazy, których nie ma dzielić pomiędzy linie

%\usepackage{showidx}  % do debugowania indeksu
\usepackage{makeidx}   % do robienia indeksu
\makeindex             % zrób indeks

\newcommand{\cudzyslow}[1]{,,#1''}       % zwyczajowy czydzysłów dla polskiego
\newcommand{\cytuj}[1]{ \cite{#1}}       % polski alias na komendę
\newcommand{\przypis}[1]{\footnote{#1}}  % polski alias na komendę
\newcommand{\indeks}{\index}             % polski alias na komendę

\newcommand{\produkt}[1]{\textsc{#1}}

% środowisko do cytatów
\newenvironment{cytat}%
{\begin{quote}}%
{\end{quote}}

% w razie pracy matematycznej
%\newtheorem{warunek}{Warunek}
%
%\newtheorem{definicjaa}{Definicja}
%\newenvironment{definicja}[1][]%
%{\begin{definicjaa}[#1]}%
%{\end{definicjaa}}
%
%\newtheorem{twierdzenie}{Twierdzenie}
%\newtheorem{lemat}{Lemat}

% środowisko do przykładu
%\newenvironment{przyklad}[1][Przykład.]%
%{\begin{trivlist}\item[\hskip \labelsep {\bfseries #1}]}%
%{\end{trivlist}}

% dowód tj. przykład
%\newenvironment{dowod}[1][Dowód.]%
%{\begin{przyklad}[#1]}%
%{\end{przyklad}}

% wniosek z tw. też jak przykład
%\newenvironment{wniosek}[1][Wniosek.]%
%{\begin{przyklad}[#1]}%
%{\end{przyklad}}


\begin{document}
\selectlanguage{polish}

% odręcza strona tytułowa
\begin{titlepage}
\begin{center}
{\Large Nazwa Szkoły}\\[16em]

\textsf{\huge Imię i Nazwisko}\\[1em]
\textsf{\Huge Tytuł pracy\\który nie mieści się w jednej linii}\\[2em]
{\Large Praca inżynierska}\\[1em]
{\Large Promotor: stopień naukowy Imię Nazwisko}

\vfill

{\Large MIEJSCE ZŁOŻENIA 2011}
\end{center}
\end{titlepage}

\addtocounter{page}{1}     % strona tytułowa to też strona i trzeba ją liczyć
\setcounter{tocdepth}{3}   % spis treści na 3 poziomy
\tableofcontents           % zrób spis

% pliki do zaczytania
\chapter{Wprowadzenie}

To jest najważniejszy kawałek pracy --- to streszczenie i przewodnik.
Zaczyna się od tego o co w ogóle chodzi.

Później jest akapit o tym co jest celem tej pracy.

A jeszcze później kilka akapitów, które mówią co jest, w którym rozdziale.
(Że np. w rozdziale \ref{teoria} jest wszystko co trzeba wiedzieć, żeby zrozumieć 
to, co jest w części praktycznej, a więc w rozdziale \ref{praktyka}).
          
\chapter{Teoria}

\section{Wykorzystane technologie}
\index{metoda!fajna}

A to jej omówienie

\subsection{Podzielone na}

fragmenty



\chapter{Praktyka}
\label{praktyka} 

A oto mój super projekt w całej swej okazałości.

\chapter{Podsumowanie}

Fajnie było, ale trzeba jeszcze zrobić to i owo, żeby było super.



\bibliography{literatura}  % bibliografia

\cleardoublepage     % magia
\phantomsection      % magia
\listoffigures       % lista rysunków

\cleardoublepage     % magia
\phantomsection      % magia
%\addcontentsline{toc}{chapter}{Indeks}   % indeksu nie umiesza się w spisie treści, ale jeśli ktoś musi...
\printindex          % wstaw indeks

\end{document}
